\documentclass{ltxguidex}
\usepackage{fontspec}
\setmainfont{Tiempos Text}
\usepackage{FiraSans}
\usepackage{FiraMono}

\newcommand{\reqopt}[1]{Requires the \texttt{\color{magenta}#1} option.\par}
\usepackage{hologo}
\usepackage[quote]{rebeccastyle}
\newacronyms{mla,}
\newcommand{\fontspecok}{\hologo{XeLaTeX} or \hologo{LuaTeX}}
\newcommand{\re}{\ltxclass{rebeccaessay}}
\newcommand{\rs}{\package{rebeccastyle}}
\title{The \re\ document class}
\author{Rebecca Turner\thanks{Brandeis University; \email{rebeccaturner@brandeis.edu}}}
\date{2018-10-20}
\begin{document}
\maketitle

\begin{abstract}
	\re\ is a document class extending \ltxclass{article} for
	\fontspecok\ typeset in Garamond, with \mla\ citations via
	\hologo{BibTeX}, variable line spacing, and a |coverletter|
	environment.

	\re\ is, as the name implies, mostly intended for personal use.
\end{abstract}

\tableofcontents
\vfill
\pagebreak

\section{Package options}

By default, \rs\ does very little; additional features are opted into with
package options. Unless otherwise noted, an option can be explicitly
declined by prefixing it with |no|; for example, |bib| is disabled with
|nobib|.

\begin{keys}
	\key{tabu}
	Loads table packages \ctan{multirow}, \ctan{booktabs}, and
	\ctan{tabu}.

	Makes the |\Th| command available.

	\key{figure}
	Loads \ctan{graphicx} and \ctan{caption}, makes figure captions
	prettier.

	\key{list}
	Loads \ctan{enumitem}.

	\key{footnote}
	Loads \ctan{footmisc} and makes footnotes prettier.

	\key{quote}
	Redefines the |quote| and |quotation| environments to add an
	optional argument.

	\key{bib}
	Sets up the document for an \mla-cited bibliography using Biber.
	Loads \ctan{biblatex}, \ctan{csquotes}, and \ctan{babel}.

	Makes the |\fullcite| command available.

	\key{font}
	Loads \ctan{fontspec} and \ctan{gillius}.

	\key{times}
	Sets the font to Times New Roman.

	\key{garamond}
	Sets the font to Garamond Premier Pro. Note that \option{garamond}
	is the default when the \option{font} option is given.

	\key{1}
	\key{1.5}
	\key{2}
	Controls leading with the \ctan{setspace} package. These options
	have no negative forms (e.x.\ \option{no1}).

	\begin{note}
		The \option{1} option differs from the default in that it
		loads the \package{setspace} package and calls
		|\singlespace|.
	\end{note}
\end{keys}

\subsection{Default behavior}

\rs\ does perform some configuration regardless of what options are passed.
This includes:

\begin{itemize}
	\item Loading \ctan{geometry} and setting 1-inch margins.
	\item Loading \ctan{hyperref}.
	\item Loading \ctan{fancyhdr} and setting a better page-header.
	\item Loading \ctan{titlesec} and setting better formatting for
	|\section| and cousins.
	\item Making |\maketitle| better.
\end{itemize}

\section{The \re\ document class}
Using the \re\ document class is mostly equivalent to
\begin{latexcode}
\documentclass[12pt]{article}
\usepackage[fonts, footnote, quote, bib]{rebeccastyle}
\end{latexcode}

However, the default options can be overridden; passing \option{nobib} will
result in no bibliography system being loaded.

Unknown class options are passed to \ltxclass{article}.

\section{Commands}

\begin{desc}
|\newacronym[<command>]{<acronym text>}|
\end{desc}

Creates a new acronym. If |<command>| isn't given, the text of the macro
will be used instead; |\newacronym{mla}| would create |\mla|. If the
resulting command already exists, this will explode.

\begin{LTXexample}
\newacronym[\xyz]{mla}
\xyz

\newacronym{ussr}
The \ussr\ was\dots
\end{LTXexample}

\begin{desc}
|\Th[<column spec>]{<header text>}|
\end{desc}

\reqopt{tabu} Typesets a table header in bold-face. |<column spec>| defaults
to |l|. Useful for when a column is wrapped in a math environment.

\begin{desc}
|\begin{quote}[<options>]...\end{quote}|\\
|\begin{quotation}[<options>]...\end{quotation}|\\
\end{desc}

\reqopt{quote} A wrapper around the original |quote| and |quotation|
environments which adds support for sources. |<options>| may include:
\begin{keys}
	\key{author}[\m{author}]
	The quote's author.

	\key{source}[\m{source}]
	The quote source. If the \option{author} option is also given, the
	source is printed in parenthesis.

	\key{cite}[\m{bib key}]
	A citation key. If given, |\textcite{<bib key>}| is used for the
	attribution. If given with \option{source} or \option{author}, may
	lead to unexpected results.
\end{keys}

\begin{desc}
|\fullcite{<bib key>}|
\end{desc}

\reqopt{bib} Prints a full citation as it will appear in the bibliography.

\end{document}
